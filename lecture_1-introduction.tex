% !TeX program = lualatex

\documentclass[smaller,xcolor=table,aspectratio=169]{beamer}

\mode<presentation>
{
\usetheme{PrincetonJunction}
\setbeamercovered{transparent}
}

% - These packages are used because Mathematica's "official" font is Arial,
%   which is a TrueType font.  Unfortunately, TrueType fonts cannot be used
%   with vanilla latex (or latexpdf), so you need to use either xetex or,
%   better, lualatex to compile this.  Alternately, you can comment out these
%   lines, add a usepackage command for babel, and go from there.  Just don't
%   be sad when you get in trouble with the Communications department.

\title[Short Title]{Title of Presentation}

\subtitle{Subtitle}

% - Use the \inst{?} command only if the authors have different
%   affiliation.
\author{Author \and Author \and Author\\Author \and Author \and Author}

% - Use the \inst command only if there are several affiliations.
% - Keep it simple, no one is interested in your street address.
\institute[Universities of]
{
\inst{1}Department of Computer Science, Univ of S
\and
\inst{2}Department of Theoretical Philosophy, Univ of E
}

% Should also contain the venue
\date{Presentation at the XXX Conference\\
City, State\\
Date 201x}


% This is only inserted into the PDF information catalog. Can be left
% out.
\subject{Talks}



% Then you can add a logo as follows:
\pgfdeclareimage[height=\beamerheadheight]{big-m}{logos/m-impact_blue_rays.pdf}
\titlegraphic{\pgfuseimage{big-m}}
\pgfdeclareimage[height=18pt]{mathematica-logo}{logos/m-impact_blue_rays.pdf}
\logo{\pgfuseimage{mathematica-logo}}

% Delete this if you do not want the table of contents to pop up at
% the beginning of each subsection:
\AtBeginSubsection[]
{
\begin{frame}<beamer>
\frametitle{Outline}
\tableofcontents[currentsection,currentsubsection]
\end{frame}
}

% If you wish to uncover everything in a step-wise fashion, uncomment
% the following command:
%\beamerdefaultoverlayspecification{<+->}

\begin{document}

{%
\logo{\rule{0pt}{\beamerfootheight}}%
\begin{frame}[noframenumbering]%
\titlepage%
\end{frame}%
}

%\begin{frame}
%\frametitle{Outline}
%\tableofcontents
% You might wish to add the option [pausesections]
%\end{frame}


\section{Introduction to Natural Language Processing}

\begin{frame}[plain]
\begin{centering}
  Natural Language Processing
  for the purposes of this course
  Converting written human communications into quantitative data
\end{centering}
\end{frame}

\begin{frame}
  \frametitle{Examples}
  \begin{itemize}
    \item Professional Notes
    \begin{itemize}
      \item Doctor's notes (Doctor \(\Rightarrow\) Doctor)
      \item Meeting minutes (Clerk \(\Rightarrow\) Public)
      \item Shorthand scrawl (Reporter \(\Rightarrow\) Self)
    \end{itemize}
    \item Laws and Policies
    \begin{itemize}
      \item Legislation
      \item Executive orders
      \item Policy rules (Federal Register)
      \item Court rulings
    \end{itemize}
    \item Interview transcriptions
    \item Academic studies
    \item Historical literature
    \item Wikipedia articles
    \item Random websites
  \end{itemize}
\end{frame}

\begin{frame}
  \frametitle{Revelance to MPR}
  Doctor's notes:  understanding a Medicare patient's condition to better predict the outcome of a particular surgical procedure
  Policy rules:  collecting different states' policies concerning a particular social service (e.g., child abuse prevention) to compare and contrast implementation
  Interview transcripts:  extracting and highlighting common topics from thousands of on-site interviews to evaluate policy understanding by practitioners
  Academic studies:  detect and gather high-quality articles for systematic review
\end{frame}

\begin{frame}

\end{frame}

\section*{Summary}

\begin{frame}
\frametitle<presentation>{Summary}

\begin{itemize}
  \item The \alert{first main message} of your talk in one or two lines.
\end{itemize}

% The following outlook is optional.
\vfill
\begin{itemize}
  \item Outlook
  \begin{itemize}
    \item Something you haven't solved.
    \item Something else you haven't solved.
  \end{itemize}
\end{itemize}
\vfill
\end{frame}

\end{document}
