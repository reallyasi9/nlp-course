% !TeX program = lualatex

\documentclass[smaller,xcolor=table,aspectratio=169]{beamer}

\usepackage{listings}

\mode<presentation>
{
\usetheme{PrincetonJunction}
\setbeamercovered{transparent}
}

% - These packages are used because Mathematica's "official" font is Arial,
%   which is a TrueType font.  Unfortunately, TrueType fonts cannot be used
%   with vanilla latex (or latexpdf), so you need to use either xetex or,
%   better, lualatex to compile this.  Alternately, you can comment out these
%   lines, add a usepackage command for babel, and go from there.  Just don't
%   be sad when you get in trouble with the Communications department.

\title[Short Title]{Title of Presentation}

\subtitle{Subtitle}

% - Use the \inst{?} command only if the authors have different
%   affiliation.
\author{Phil Killewald}

% - Use the \inst command only if there are several affiliations.
% - Keep it simple, no one is interested in your street address.
% \institute[Universities of]
% {
% \inst{1}Department of Computer Science, Univ of S
% \and
% \inst{2}Department of Theoretical Philosophy, Univ of E
% }

% Should also contain the venue
\venue{Presentation at the XXX Conference}
\location{City, State}
\date{\today}


% This is only inserted into the PDF information catalog. Can be left
% out.
\subject{Talks}



% Then you can add a logo as follows:
\pgfdeclareimage[height=5cm]{big-m}{logos/m-impact_blue_rays.pdf}
\titlegraphic{\pgfuseimage{big-m}}
\pgfdeclareimage[height=18pt]{mathematica-logo}{logos/m-impact_blue_rays.pdf}
\logo{\pgfuseimage{mathematica-logo}}

% Delete this if you do not want the table of contents to pop up at
% the beginning of each subsection:
\AtBeginSubsection[]
{
\begin{frame}<beamer>
\frametitle{Outline}
\tableofcontents[currentsection,currentsubsection]
\end{frame}
}

% If you wish to uncover everything in a step-wise fashion, uncomment
% the following command:
%\beamerdefaultoverlayspecification{<+->}

\begin{document}

{%
\logo{\rule{0pt}{\beamerfootheight}}%
\begin{frame}[noframenumbering]%
\titlepage%
\end{frame}%
}

%\begin{frame}
%\frametitle{Outline}
%\tableofcontents
% You might wish to add the option [pausesections]
%\end{frame}


\section{Introduction to Natural Language Processing}

\begin{frame}[plain]
\begin{centering}
  Natural Language Processing
  for the purposes of this course
  Converting written human communications into quantitative data
\end{centering}
\end{frame}

\begin{frame}
  \frametitle{Examples}
  \begin{itemize}
    \item Professional Notes
    \begin{itemize}
      \item Doctor's notes (Doctor \(\Rightarrow\) Doctor)
      \item Meeting minutes (Clerk \(\Rightarrow\) Public)
      \item Shorthand scrawl (Reporter \(\Rightarrow\) Self)
    \end{itemize}
    \item Laws and Policies
    \begin{itemize}
      \item Legislation
      \item Executive orders
      \item Policy rules (Federal Register)
      \item Court rulings
    \end{itemize}
    \item Interview transcriptions
    \item Academic studies
    \item Historical literature
    \item Wikipedia articles
    \item Random websites
  \end{itemize}
\end{frame}

\begin{frame}
  \frametitle{Revelance to MPR}
  Doctor's notes:  understanding a Medicare patient's condition to better predict the outcome of a particular surgical procedure
  Case worker visitation notes:  predicting job search outcomes using written descriptions of home life
  Policy rules:  collecting different states' policies concerning a particular social service (e.g., child abuse prevention) to compare and contrast implementation
  Interview transcripts:  extracting and highlighting common topics from thousands of on-site interviews to evaluate policy understanding by practitioners
  Academic studies:  detecting and gathering high-quality articles for systematic review
\end{frame}

\begin{frame}
  \frametitle{Course Timeline}
  \begin{enumerate}
    \item Course Introduction
    \item Computer Representation of Text \& Canned Libraries
    \item Preparing Text
    \item Traditional NLP
    \item Quantification of Text
    \item Modern Quantification of Text
    \item State-of-the-Art Quantification of Text
  \end{enumerate}
\end{frame}

\section{Computer Representations of Text}

\begin{frame}
  \frametitle{Bytes}
  Computers understand bytes.
  Byte: an 8-digit binary number
  Bytes can take values from 0 to 255.
  Everything a computer does is based around bytes:
  Integer math
  Booleans
  Floating point operations
  File reading/writing
  Literally everything.
  But people understand glyphs
  I mean, people understand numbers as well, but those are pictoral representations of concepts.  People read and understand the pictures.
\end{frame}

\begin{frame}
  \frametitle{ASCII}
  Ancient computers communicated everything using bytes.
  Picture of ENIAC
  This was annoying.
\end{frame}

\begin{frame}[plain]
  (American) computer scientists developed a way for computers to represent text for input and output purposes.
  ASCII: American Standard Code for Information Interchange
  Standard that encodes numbers 0 through 127 as Latin glyphs.
  (table)
  65 = A
  90 = Z
  97 = a
  122 = z
  33 = !
  63 = ?
  32 = (space)
  10 = (line feed)
  6 = (acknowledgement)
\end{frame}

\begin{frame}[plain]
  Problems with ASCII:
  Other languages use more glyphs than those in American English:
  £
  ¿
  á
  «»
  Temporary solution: use that 8th bit!
  (add ISO 8859-1 to table)
\end{frame}

\begin{frame}[plain]
  Windows:  Looks like you forgot some curley quote marks, so I fixed it for you.
  (add Windows-1252)
\end{frame}

\newfontface\hebfont{NotoSansHebrew}[Scale=MatchLowercase]
\newfontface\chkfont{NotoSansCherokee}[Scale=MatchLowercase]
\newfontface\chnfont{NotoSansCJKsc}[Scale=MatchLowercase]
\newcommand{\hebtxt}[1]% Hebrew text inside LTR
    {\bgroup\textdir TRT\hebfont #1\egroup}

\begin{frame}
  \frametitle{Problems}
  Turns out there are languages out there other than those that use forms of the Latin alphabet.
  \begin{itemize}
    \item hello world
    \item Здравствуй, мир (ru)
    \item \hebtxt{שלום עולם} (he)
    \item {\chkfont ᎣᏏᏲ ᎡᎶᎯ} (ck)
    \item {\chnfont 你好世界} (cn)
  \end{itemize}
\end{frame}

\begin{frame}
  \frametitle{Unicode}
  Solution:  Variable-width encoding (specifically, UTF-8)
  Advantages:
  \begin{itemize}
    \item Large number of symbols ($17$ {\em planes} of $2^{16}$ {\em code points})
    \item Backwards-compatable with ASCII (7-bit) and some extended ASCII (8-bit) symbols
  \end{itemize}
  Disadvantages:
  \begin{itemize}
    \item Lengths of strings are hard
    \begin{itemize}
      \item Number of bytes != number of code points
      \item Number of code points != number of glyphs
    \end{itemize}
    \item Microsoft is difficult
    \begin{itemize}
      \item Curly quotes, elipses, dashes, bullets, etc. in {\em continuation code} space of UTF-8
    \end{itemize}
  \end{itemize}
\end{frame}

\newfontface\emojifont{NotoEmoji-Regular}
\begin{frame}
  \frametitle{Examples}
  Hi, world!
  48 69 2c 20 77 6f 72 6c 64 21

  Hi, {\emojifont 🌎}!
  48 69 2c 20 f09f8c8e 21

  Hi, {\emojifont 🇺🇸}!
  48 69 2c 20 f09f87ba f09f87b8 21
\end{frame}

\begin{frame}
  \frametitle{Getting It Wrong}
  \begin{enumerate}
    \item Using the wrong encoding can make text undecypherable.
    \item Encoding information is out-of-band--you must be told what encoding to use.
    \item Nobody will have any idea what you are talking about.
  \end{enumerate}

  48 69 2c 20 f0 9f 87 ba f0 9f 87 b8 21
  UTF-8: Hi, {\emojifont 🇺🇸}!
  Windows-1252: Hi, 🇰🇸!
\end{frame}

\lstset{language=Python}
\lstset{
  morekeywords={with,as},
  morekeywords={[2]>>>}
}
\lstdefinestyle{mystyle}{
  basicstyle=\footnotesize\color{black}\ttfamily,
  backgroundcolor=\color{mathematica light neutral},
  showspaces=false,
  showtabs=false,
  breaklines=true,
  showstringspaces=false,
  breakatwhitespace=true,
  % Color settings
  keywordstyle=\color{mathematica brown}\bfseries, % core keywords
  keywordstyle={[2]\color{mathematica dark purple}\bfseries}, % built-ins
  keywordstyle={[3]\color{mathematica neutral}\bfseries}, % built-ins
  stringstyle=\color{mathematica dark green},
  commentstyle=\color{mathematica dark neutral}\itshape,
  upquote=true
}

\begin{frame}[fragile]
  \frametitle{Guessing (Intelligently)}
  chardet
  \begin{lstlisting}[style=mystyle]
>>> import urllib.request
>>> import chardet
>>> with urllib.request.urlopen('http://google.co.jp') as req:
>>>     rawdata = req.read()
>>>     for x in rawdata:
>>>         pass
>>>     chardet.detect(rawdata)
{'encoding': 'Windows-1252', 'confidence': 0.73, 'language': ''}
  \end{lstlisting}

\end{frame}

\section*{Summary}

\begin{frame}
\frametitle<presentation>{Summary}

\begin{itemize}
  \item The \alert{first main message} of your talk in one or two lines.
\end{itemize}

% The following outlook is optional.
\vfill
\begin{itemize}
  \item Outlook
  \begin{itemize}
    \item Something you haven't solved.
    \item Something else you haven't solved.
  \end{itemize}
\end{itemize}
\vfill
\end{frame}

\end{document}
