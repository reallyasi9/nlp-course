% !TeX program = lualatex

\documentclass[smaller,xcolor=table,aspectratio=169]{beamer}

\usepackage{listings}
\usepackage[table]{xcolor}

% Useful for writing out a space character
% See https://tex.stackexchange.com/a/50807
\newcommand\Vtextvisiblespace[1][.3em]{%
  \mbox{\kern.06em\vrule height.3ex}%
  \vbox{\hrule width#1}%
  \hbox{\vrule height.3ex}}

\mode<presentation>
{
\usetheme{PrincetonJunction}
\setbeamercovered{invisible}
}

% - These packages are used because Mathematica's "official" font is Arial,
%   which is a TrueType font.  Unfortunately, TrueType fonts cannot be used
%   with vanilla latex (or latexpdf), so you need to use either xetex or,
%   better, lualatex to compile this.  Alternately, you can comment out these
%   lines, add a usepackage command for babel, and go from there.  Just don't
%   be sad when you get in trouble with the Communications department.

\title[NLP Course 1]{Natural Language Processing}

\subtitle{Text, Numbers, and Computers}

% - Use the \inst{?} command only if the authors have different
%   affiliation.
\author{Phil Killewald}

% - Use the \inst command only if there are several affiliations.
% - Keep it simple, no one is interested in your street address.
% \institute[Universities of]
% {
% \inst{1}Department of Computer Science, Univ of S
% \and
% \inst{2}Department of Theoretical Philosophy, Univ of E
% }

% Should also contain the venue
\venue{Part of the Mathematica Advanced Analytics Training Series}
\location{Course 1 of N}
\date{Originally written in August of 2018}

% This is only inserted into the PDF information catalog. Can be left
% out.
\subject{Courses}



% Then you can add a logo as follows:
\pgfdeclareimage[height=5cm]{big-m}{logos/m-impact_blue_rays.pdf}
\titlegraphic{\pgfuseimage{big-m}}
\pgfdeclareimage[height=18pt]{mathematica-logo}{logos/m-impact_blue_rays.pdf}
\logo{\pgfuseimage{mathematica-logo}}

% Delete this if you do not want the table of contents to pop up at
% the beginning of each subsection:
\AtBeginSubsection[]
{
\begin{frame}<beamer>
\frametitle{Outline}
\tableofcontents[currentsection,currentsubsection]
\end{frame}
}

% If you wish to uncover everything in a step-wise fashion, uncomment
% the following command:
%\beamerdefaultoverlayspecification{<+->}

\begin{document}

{%
\logo{\rule{0pt}{\beamerfootheight}}%
\begin{frame}[noframenumbering]%
\titlepage%
\end{frame}%
}

%\begin{frame}
%\frametitle{Outline}
%\tableofcontents
% You might wish to add the option [pausesections]
%\end{frame}


\section{Introduction to Natural Language Processing}

\begin{frame}[plain]
\begin{centering}
  Natural Language Processing\\
  (for the purposes of this course)\\
  Converting written human communications into quantitative data
\end{centering}
\end{frame}

\begin{frame}
  \frametitle{Examples}
  \begin{itemize}
    \item Professional Notes
    \begin{itemize}
      \item Doctor's notes (Doctor \(\Rightarrow\) Doctor)
      \item Meeting minutes (Clerk \(\Rightarrow\) Public)
      \item Shorthand scrawl (Reporter \(\Rightarrow\) Self)
    \end{itemize}
    \item Laws and Policies
    \begin{itemize}
      \item Legislation
      \item Executive orders
      \item Policy rules (Federal Register)
      \item Court rulings
    \end{itemize}
    \item Interview transcriptions
    \item Academic studies
    \item Historical literature
    \item Wikipedia articles
    \item Random websites
  \end{itemize}
\end{frame}

\begin{frame}
  \frametitle{Revelance to MPR}
  Doctor's notes:  understanding a Medicare patient's condition to better predict the outcome of a particular surgical procedure\\
  Case worker visitation notes:  predicting job search outcomes using written descriptions of home life\\
  Policy rules:  collecting different states' policies concerning a particular social service (e.g., child abuse prevention) to compare and contrast implementation\\
  Interview transcripts:  extracting and highlighting common topics from thousands of on-site interviews to evaluate policy understanding by practitioners\\
  Academic studies:  detecting and gathering high-quality articles for systematic review
  \vfill{}
  Note: None of these are designed explicitly to communicate quantifiable information!
\end{frame}

\begin{frame}
  \frametitle{Course Timeline}
  \begin{enumerate}
    \item Course Introduction
    \item Computer Representation of Text \& Canned Libraries
    \item Preparing Text
    \item Traditional NLP
    \item Quantification of Text
    \item Modern Quantification of Text
    \item State-of-the-Art Quantification of Text
  \end{enumerate}
\end{frame}

\section{Computer Representations of Text}

\begin{frame}
  \frametitle{Bytes}
  Computers understand bytes.\\\pause{}
  Byte: an 8-digit binary number\\\pause{}
  Bytes can take values from 0 to 255.\\\pause{}
  Everything a computer does is based around bytes:\\\pause{}
  \begin{itemize}
    \item Integer math
    \item Booleans
    \item Floating point operations
    \item File reading/writing
    \item Literally everything
  \end{itemize}
  But people understand glyphs
\end{frame}

\begin{frame}
  \frametitle{ASCII}
  Ancient computers communicated everything using bytes.\\
  \includegraphics[width=.5\textwidth]{images/eniac.jpg}\pause{}\\
  This was annoying.
\end{frame}

\newcommand{\CNT}{\cellcolor{mathematica green!25}}
\newcommand{\ALP}{\cellcolor{mathematica orange!25}}
\newcommand{\PUN}{\cellcolor{mathematica purple!25}}
\begin{frame}[plain]
  (American) computer scientists developed a way for computers to represent text for input and output purposes.\\
  ASCII: American Standard Code for Information Interchange\\
  Standard that encodes numbers 0 through 127 (0x00 through 0x7F) as Latin glyphs.\\
  \begin{table}
    \tiny\ttfamily
    \begin{tabular}{r|c|c|c|c|c|c|c|c|c|c|c|c|c|c|c|c}
      & \_0 & \_1 & \_2 & \_3 & \_4 & \_5 & \_6 & \_7 & \_8 & \_9 & \_A & \_B & \_C & \_D & \_E & \_F \\ \hline
      0\_ & \CNT nul & \CNT soh & \CNT stx & \CNT etx & \CNT eot & \CNT enq & \CNT ack & \CNT bel & \CNT bs & \CNT ht & \CNT lf & \CNT vt & \CNT ff & \CNT cr & \CNT so & \CNT si \\
      1\_ & \CNT dle & \CNT dc1 & \CNT dc2 & \CNT dc3 & \CNT dc4 & \CNT nak & \CNT syn & \CNT etb & \CNT can & \CNT em & \CNT sub & \CNT esc & \CNT fs & \CNT gs & \CNT rs & \CNT us \\
      2\_ & \PUN \Vtextvisiblespace & \PUN ! & \PUN " & \PUN \# & \PUN \$ & \PUN \% & \PUN \& & \PUN ' & \PUN ( & \PUN ) & \PUN * & \PUN + & \PUN , & \PUN - & \PUN . & \PUN / \\
      3\_ & \ALP 0 & \ALP 1 & \ALP 2 & \ALP 3 & \ALP 4 & \ALP 5 & \ALP 6 & \ALP 7 & \ALP 8 & \ALP 9 & \PUN : & \PUN ; & \PUN < & \PUN = & \PUN > & \PUN ? \\
      4\_ & \PUN @ & \ALP A & \ALP B & \ALP C & \ALP D & \ALP E & \ALP F & \ALP G & \ALP H & \ALP I & \ALP J & \ALP K & \ALP L & \ALP M & \ALP N & \ALP O \\
      5\_ & \ALP P & \ALP Q & \ALP R & \ALP S & \ALP T & \ALP U & \ALP V & \ALP W & \ALP X & \ALP Y & \ALP Z & \PUN [ & \PUN \textbackslash & \PUN ] & \PUN \textasciicircum & \PUN \_ \\
      6\_ & \PUN ` & \ALP a & \ALP b & \ALP c & \ALP d & \ALP e & \ALP f & \ALP g & \ALP h & \ALP i & \ALP j & \ALP k & \ALP l & \ALP m & \ALP n & \ALP o \\
      7\_ & \ALP p & \ALP q & \ALP r & \ALP s & \ALP t & \ALP u & \ALP v & \ALP w & \ALP x & \ALP y & \ALP z & \PUN \{ & \PUN | & \PUN \} & \PUN \textasciitilde & \CNT del \\
    \end{tabular}
  \end{table}
\end{frame}

\newcommand{\UND}{\cellcolor{black}\color{white}}
\begin{frame}[plain]
  Problems with ASCII:\\\pause{}
  Other languages use more glyphs than those in American English:\\\pause{}
  \begin{itemize}
    \item £
    \item ¿
    \item á
    \item «»
  \end{itemize}\pause{}
  Temporary solution: use that 8th bit!\\
  \begin{table}
    \footnotesize
    \begin{tabular}{r|c|c|c|c|c|c|c|c|c|c|c|c|c|c|c|c}
      & \_0 & \_1 & \_2 & \_3 & \_4 & \_5 & \_6 & \_7 & \_8 & \_9 & \_A & \_B & \_C & \_D & \_E & \_F \\ \hline
      8\_ & \UND {\visible<5->€} & \UND & \UND {\visible<5->‚} & \UND {\visible<5->ƒ} & \UND {\visible<5->„} & \UND {\visible<5->…} & \UND {\visible<5->†} & \UND {\visible<5->‡} & \UND {\visible<5->ˆ} & \UND {\visible<5->‰} & \UND {\visible<5->Š} & \UND {\visible<5->‹} & \UND {\visible<5->Œ} & \UND & \UND {\visible<5->Ž} & \UND \\
      9\_ & \UND & \UND {\visible<5->‘} & \UND {\visible<5->’} & \UND {\visible<5->“} & \UND {\visible<5->”} & \UND {\visible<5->•} & \UND {\visible<5->–} & \UND {\visible<5->—} & \UND {\visible<5->˜} & \UND {\visible<5->™} & \UND {\visible<5->š} & \UND {\visible<5->›} & \UND {\visible<5->œ} & \UND & \UND {\visible<5->ž} & \UND {\visible<5->Ÿ} \\
      A\_ & \PUN nbsp & \PUN ¡ & \PUN ¢ & \PUN £ & \PUN ¤ & \PUN ¥ & \PUN ¦ & \PUN § & \PUN ¨ & \PUN © & \PUN ª & \PUN « & \PUN ¬ & \PUN shy & \PUN ® & \PUN ¯ \\
      B\_ & \PUN ° & \PUN ± & \ALP ² & \ALP ³ & \PUN ´ & \ALP µ & \PUN ¶ & \PUN · & \PUN ¸ & \ALP ¹ & \PUN º & \PUN » & \ALP ¼ & \ALP ½ & \ALP ¾ & \PUN ¿ \\
      C\_ & \ALP À & \ALP Á & \ALP Â & \ALP Ã & \ALP Ä & \ALP Å & \ALP Æ & \ALP Ç & \ALP È & \ALP É & \ALP Ê & \ALP Ë & \ALP Ì & \ALP Í & \ALP Î & \ALP Ï \\
      D\_ & \ALP Ð & \ALP Ñ & \ALP Ò & \ALP Ó & \ALP Ô & \ALP Õ & \ALP Ö & \PUN × & \ALP Ø & \ALP Ù & \ALP Ú & \ALP Û & \ALP Ü & \ALP Ý & \ALP Þ & \ALP ß \\
      E\_ & \ALP à & \ALP á & \ALP â & \ALP ã & \ALP ä & \ALP å & \ALP æ & \ALP ç & \ALP è & \ALP é & \ALP ê & \ALP ë & \ALP ì & \ALP í & \ALP î & \ALP ï \\
      F\_ & \ALP ð & \ALP ñ & \ALP ò & \ALP ó & \ALP ô & \ALP õ & \ALP ö & \PUN ÷ & \ALP ø & \ALP ù & \ALP ú & \ALP û & \ALP ü & \ALP ý & \ALP þ & \ALP ÿ \\
    \end{tabular}
  \end{table}
  {\visible<5->Windows:  Looks like you forgot some curley quote marks, so I fixed it for you. (Windows-1252)}
\end{frame}

\newfontface\hebfont{NotoSansHebrew}[Scale=MatchLowercase]
\newfontface\chkfont{NotoSansCherokee}[Scale=MatchLowercase]
\newfontface\chnfont{NotoSansCJKsc}[Scale=MatchLowercase]
\newcommand{\hebtxt}[1]% Hebrew text inside LTR
    {\bgroup\textdir TRT\hebfont #1\egroup}

\begin{frame}
  \frametitle{Problems}
  Turns out there are languages out there other than those that use forms of the Latin alphabet.
  \begin{itemize}
    \item<1-> hello world
    \item<2-> Здравствуй, мир (ru)
    \item<3-> {\chkfont ᎣᏏᏲ ᎡᎶᎯ} (ck)
    \item<4-> {\chnfont 你好世界} (cn)
  \end{itemize}
\end{frame}

\begin{frame}
  \frametitle{Unicode}
  Solution:  Variable-width encoding (specifically, UTF-8)
  Advantages:
  \begin{itemize}
    \item Large number of symbols ($17$ {\em planes} of $2^{16}$ {\em code points})
    \item Backwards-compatable with ASCII (7-bit) and some extended ASCII (8-bit) symbols
  \end{itemize}
  Disadvantages:
  \begin{itemize}
    \item Lengths of strings are hard
    \begin{itemize}
      \item Number of bytes != number of code points
      \item Number of code points != number of glyphs
    \end{itemize}
    \item Microsoft is difficult
    \begin{itemize}
      \item Curly quotes, elipses, dashes, bullets, etc. in {\em continuation code} space of UTF-8
    \end{itemize}
  \end{itemize}
\end{frame}

\newfontface\emojifont{NotoEmoji-Regular}
\begin{frame}
  \frametitle{Examples}
  \begin{table}
    \small
    \begin{tabular}{|c|c|c|c|c|c|c|c|c|}
      \hline
      48 & 69 & 2c & 20                & 77 & 6f & 72 & 6c & 64 \\ \hline
      H  & i  & ,  & \Vtextvisiblespace & w  & o  & r  & l  & d \\ \hline
    \end{tabular}
  \end{table}

  \begin{table}
    \small
    \begin{tabular}{|c c c|c c c|c c c|c c c|}
      \hline
      e4 & bd & a0 & e5 & a5 & bd & e4 & b8 & 96 & e7 & 95 & 8c \\ \hline
      \multicolumn{3}{|c|}{\chnfont 你} & \multicolumn{3}{c|}{\chnfont 好} & \multicolumn{3}{c|}{\chnfont 世} & \multicolumn{3}{c|}{\chnfont 界} \\ \hline
    \end{tabular}
  \end{table}

  \begin{table}
    \small
    \begin{tabular}{|c|c|c|c|c c c c|c c c c|}
      \hline
      48 & 69 & 2c & 20 & f0 & 9f & 87 & ba & f0 & 9f & 87 & b8 \\ \hline
      H & i & , & \Vtextvisiblespace & \multicolumn{4}{c|}{\emojifont 🇺} & \multicolumn{4}{c|}{\emojifont 🇸} \\ \hline
      H & i & , & \Vtextvisiblespace & \multicolumn{8}{c|}{\emojifont 🇺🇸} \\ \hline
    \end{tabular}
  \end{table}

  \begin{table}
    \small
    \begin{tabular}{|c|c|c c c|c|c|c|c|c|c|c|c c c|c|c|}
      \hline
      69 & 6e & ef & ac & 81 & 6e & 69 & 74 & 65 & 20 & 77 & 61 & ef & ac & 84 & 65 & 73 \\ \hline
      i & n & \multicolumn{3}{c|}{fi} & n & i & t & e & \Vtextvisiblespace & w & a & \multicolumn{3}{c|}{ffl} & e & s \\ \hline
      i & n & \multicolumn{3}{c|}{\begin{tabular}{c c} f & i\\ \end{tabular}} & n & i & t & e & \Vtextvisiblespace & w & a & f & f & l & e & s \\ \hline
    \end{tabular}
  \end{table}

\end{frame}

\begin{frame}
  \frametitle{Getting It Wrong}
  \begin{enumerate}
    \item Using the wrong encoding can make text undecypherable.
    \item Encoding information is out-of-band--you must be told what encoding to use.
    \item Nobody will have any idea what you are talking about.
  \end{enumerate}

  \begin{table}
    \small
    \begin{tabular}{|c|c|c|c|c|c|c|c|c|c|c|c|c|}
      \hline
      48 & 69 & 2c & 20 & f0 & 9f & 87 & ba & f0 & 9f & 87 & b8 & 21 \\ \hline
      H & i & , & \Vtextvisiblespace & \multicolumn{8}{c|}{\emojifont 🇺🇸} \\ \hline
      H & i & , & \Vtextvisiblespace & ð & Ÿ & ‡ & ° & ð & Ÿ & ‡ & ¸ \\ \hline
    \end{tabular}
  \end{table}
\end{frame}

\lstset{language=Python}
\lstset{
  morekeywords={with,as},
  morekeywords=[2]{>>>}
}
\lstdefinestyle{mystyle}{
  basicstyle=\footnotesize\color{black}\ttfamily,
  backgroundcolor=\color{mathematica light neutral},
  showspaces=false,
  showtabs=false,
  breaklines=true,
  showstringspaces=false,
  breakatwhitespace=true,
  % Color settings
  keywordstyle=\color{mathematica brown}\bfseries, % core keywords
  keywordstyle={[2]\color{mathematica dark purple}\bfseries}, % built-ins
  keywordstyle={[3]\color{mathematica neutral}\bfseries}, % built-ins
  stringstyle=\color{mathematica dark green},
  commentstyle=\color{mathematica dark neutral}\itshape,
  upquote=true
}

\begin{frame}[fragile]
  \frametitle{Guessing (Intelligently)}
  chardet
  \begin{lstlisting}[style=mystyle]
>>> import urllib.request
>>> import chardet
>>> with urllib.request.urlopen('http://google.co.jp') as req:
>>>     rawdata = req.read()
>>>     for x in rawdata:
>>>         pass
>>>     chardet.detect(rawdata)
{'encoding': 'Windows-1252', 'confidence': 0.73, 'language': ''}
  \end{lstlisting}

\end{frame}

\section*{Summary}

\begin{frame}
  \frametitle{Summary}
  \begin{itemize}
    \item NLP means a lot of things
    \begin{itemize}
      \item For us at this point in time, we will focus on {\itshape quantification of written human communication}.
    \end{itemize}
    \item As a rule, human communication is not meant for quantitative analysis.
    \begin{itemize}
      \item We have to come up with ways of converting language into numbers carefully without sacraficing too much of the original meaning.
    \end{itemize}
    \item Computers only understand numbers.
    \begin{itemize}
      \item Computers represent text for input/output purposes using encodings.
      \item Encodings are one way of turning text into numbers.
      \item The universe of standard encodings is a mess, and you cannot control what encodings other people use.
    \end{itemize}
  \end{itemize}
  \begin{alertblock}{Takeaway}
    Always use UTF-8 whenever you possibly can!
  \end{alertblock}
\end{frame}

\section*{Homework}

\begin{frame}
  \frametitle{Homework}
  Encoding, Decoding, and Counting Glyphs
\end{frame}

\end{document}
